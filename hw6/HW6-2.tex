\documentclass{amsart}

% PACKAGES

\usepackage{amsmath}
\usepackage{amsfonts}
\usepackage{amssymb,enumerate}
\usepackage{amsthm,stmaryrd}
\usepackage[all]{xy}
\usepackage{hyperref}

%\theoremstyle{definition}
%\newtheorem{exer}{Exercise}

\newcommand{\bbr}{\mathbb{R}}
\newcommand{\bbc}{\mathbb{C}}
\newcommand{\bbz}{\mathbb{Z}}
\newcommand{\bbq}{\mathbb{Q}}
\newcommand{\bbn}{\mathbb{N}}
\newcommand{\be}{\mathbf{e}}
\newcommand{\ba}{\mathbf{a}}
\newcommand{\fm}{\mathfrak{m}}
\newcommand{\Hom}{\operatorname{Hom}}
\renewcommand{\ker}{\operatorname{Ker}}
\newcommand{\im}{\operatorname{Im}}
\newcommand{\xra}{\xrightarrow}
\newcommand{\wti}{\widetilde}

\theoremstyle{plain}
\newtheorem{lem}{Lemma}
\newtheorem{cor}[lem]{Corollary}
\newtheorem{prop}[lem]{Proposition}
\newtheorem{thm}[lem]{Theorem}
\newtheorem{conj}[lem]{Conjecture}
\newtheorem{intthm}{Theorem}
\renewcommand{\theintthm}{\Alph{intthm}}

\theoremstyle{definition}
\newtheorem{defn}[lem]{Definition}
\newtheorem{ex}[lem]{Example}
\newtheorem{question}[lem]{Question}
\newtheorem{questions}[lem]{Questions}
\newtheorem{problem}[lem]{Problem}
\newtheorem{disc}[lem]{Remark}
\newtheorem{rmk}[lem]{Remark}
\newtheorem{construction}[lem]{Construction}
\newtheorem{notn}[lem]{Notation}
\newtheorem{fact}[lem]{Fact}
\newtheorem{para}[lem]{}
\newtheorem{exer}[lem]{Exercise}
\newtheorem{remarkdefinition}[lem]{Remark/Definition}
\newtheorem{notation}[lem]{Notation}
\newtheorem{step}{Step}
\newtheorem{convention}[lem]{Convention}
\newtheorem*{Convention}{Convention}
\newtheorem{assumption}[lem]{Assumption}

\newcommand{\fmn}{F^{m\times n}}
\newcommand{\fnn}{F^{n\times n}}
\newcommand{\col}{\operatorname{Col}}
\newcommand{\row}{\operatorname{Row}}
\newcommand{\Span}{\operatorname{Span}}	
\newcommand{\rank}{\operatorname{rank}}	
\newcommand{\OO}[1]{\mathcal{O}_{#1}}
\begin{document}

\noindent MATH 8510, Abstract Algebra I \\
Fall 2016\\
Exercises 6-2\\
Due date Thu 29 Sep 4:00PM

\

\noindent
See the text for hints.

\

%\noindent
%Throughout this homework set, let $F$ be a field
%
%
%\

\begin{exer}[4.2.10]
Prove that every non-abelian group of order 6 has a non-normal subgroup of order 2.
Use this to classify all groups of order 6;
specifically, provide a list $G_1,G_2$ of groups of order 6 such that (1) $G_1\not\cong G_2$, and (2) for all groups $G$ if $|G|=6$, then $G\cong G_1$ or $G\cong G_2$.
Justify your answers.
\begin{proof}
Let $G$ be a group which is not abelian and $|G| = 6$.\\
Since $2 | 6$ and $3 | 6$ and $2,3$ are primes, \\
by Cauthy Theorem, there exists $H, M \leq G$ and $|H| = 2$ and $|H| = 3$.\\
Assume all the subgroups of $G$ of order 2 are normal subgroups of $G$. \\
Consider $H=\{e_G,h\} \leq G$, where $h \in G$ and $h \neq e_G$ and $h^2 = 1$. \\
Then $|H| = 2$ and $H \unlhd G$ by assumption. \\
As a result, $N_G(H) = G$.\\
So for any $g \in G$, $gH = Hg$.\\
Since $H = \{e_G,h\}$, we have $gh = hg$ for any $g \in G$.\\
So $h \in Z(G)$.\\
Since $|G| = 2\times 3$ and $2,3$ are primes, we have $G$ is abelian or $Z(G)= \{e_G\}$ by the conclusion from Exercise 1 in homework 5.\\
Given $G$ is not abelian, we have $Z(G) = \{e_G\}$, which is a contradiction since we already find $h \in Z(G)$ and $h \neq e_G$.\\
Thus, there exists a non-normal subgroup of $G$ of order 2.\\
Next we show $G$ is isomorphic to $S_3$.\\
Let $H \leq G$ of order 2 be the non-normal subgroup of $G$.\\
Then $G$ acts transitively on $G//H$. \\
Let $\pi_H: G \to S_{G//H}$ be the associated permutation representation.\\
Then
\begin{align*}
	\ker(\pi_H) &= \bigcap_{x\in G}xHx^{-1} \subset  H.
\end{align*}
So $|\ker(\pi_H)| = 1$ or $2$.\\
If $|\ker(\pi_H)| =2$, then $\ker(\pi_H) = H$, which is a contradiction since $\ker(\pi_H) \unlhd G$ and $H$ is not a normal subgroup of $G$ by assumption.\\
So $|\ker(\pi_H)| = 1$.\\
Thus, $\pi_H$ is 1-1.\\
Since $\left|G//H\right| = \frac{|G|}{|H|} = \frac{6}{2} = 3$,\\
\[ S_{G//H} \cong S_3. \]
Then $|S_{G//H}| = |S_3| = 6$.\\
Since $|G| = 6$, we have $\pi_H$ is onto since it is 1-1.\\
Thus, $\pi_H$ is bijective.\\
As a result, $\pi_H$ is an isomorphism.\\
So 
\[G \cong S_{G//H}.\]
Therefore, for each non-abelian group $G$ of order 6, we have
\[G \cong S_3. \]
Next we claim that every abelian group of order 6 is cyclic.\\
Let $G$ be a ableian group of order 6.\\
By Cauthy Theorem, there exists $x,y \in G$ and $|x| = 2, |y|= 3$.\\
Then $x \neq y$.\\
Besides, $x^{-1} = x$ and $y^{-1} = y^2 \in G$.\\
Then $x \neq y^2$, otherwise, $x=y$.\\
So we find $\{e_G,x,y,y^2\} \subset G$.\\
Similarly, we can verify that two distinct elelments $xy,xy^2 \in G$ but $xy,xy^2 \not\in \{e_G,x,y,y^2\}$.\\
Since $|G|=6$, we have
\[ G=\{e_G,x,y,y^2,xy,xy^2\}.\]
We claim $G=\langle xy \rangle$.\\
Since $G$ is abelian,
\begin{align*}
  (xy)^1 &= xy =e_G.\\
  (xy)^2 &=x^2y^2 =y^2,\\
  (xy)^3 &=x^3y^3 = x,\\
  (xy)^4 &=x^4y^4 = y,\\
  (xy)^5 &=x^5y^5 = xy^2,\\
  (xy)^6 &= x^6y^6 = e_G.
\end{align*}
Thus, $G$ is cyclic.\\
So every abelian group of order 6 is cyclic.\\
We know if a group $G$ is cyclic of order 6, then 
\[G \cong \bbz/6\bbz.\]
So for each abelian group $G$ of order 6, we have 
\[ G \cong \bbz/6\bbz.\]
Since $S_3$ is not cyclic,
\[S_3 \not\cong \bbz/6\bbz.\]
As result, we have for any non-abelian group $G$ of order 6, 
\[G \cong S_3, \] 
and for any abelian group $G$ of order 6,
\[G \cong \bbz/6\bbz,\]
where $S_3 \not\cong \bbz/6\bbz$.
\end{proof}
\end{exer}

\begin{exer}[4.3.6]
Assume that $G$ is a non-abelian group of order 15. 
Prove that $Z(G)=\{e\}$. 
Use the fact that $\langle g\rangle\leq C_G(g)$ to show that there is at most one possible class equation for $G$;
in other words, in the notation of Theorem~4.2.4 of the notes, find $r$ and $|Z(G)|$ and $[G:C_G(g_1)]$, \ldots, $[G:C_G(g_r)]$.
Justify your answers.

\begin{proof}
	Since $|G| = 15 = 3 \times 5$ and $3,5$ are primes, \\
	$G$ is abelian or $Z(G)= \{e_G\}$ by the conclusion from Exercise 1 in homework 5.\\
	Given $G$ is not abelian, we have $Z(G)=\{e_G\}$.\\
	Let $[r] = \{1,2,...,r\}$.\\
	By class equation, we have 
	\[\sum_{i=1}^r [G:C_G(g_i)] = |G|-|Z(G)| = 15-1 = 14,\]
	where $g_i \in G$ and $g_i \not\in Z(G)$ for $i\in [r]$.\\
	Then $C_G(g_i) \neq G$ for $i \in [r]$ and $g_i \neq e_G$ since $Z(G) = \{e_G\}$.\\
	So $[G:C_G(g_i)] \neq 1$.\\
	Then $[G:C_G(g_i)] \in \{3,5,15\}$ since $[G:C_G(g_i)] \mid G$.\\
	We know the fact that for $i \in [r]$, 
  	\[ \langle g_i \rangle \leq C_G(g_i).\]
  	Since $g_i \neq e_G$ for $i \in [r]$,\\
    \[C_G(g_i) \neq \{e_G\}.\]
    So for $i \in [r]$,
    \[[G:C_G(g_i)] < 15.\]
    Then for $i \in [r]$,
    \[[G:C_G(g_i)] \in \{3,5\}.\]
    So we need to find $r$ and $[G:C_G(g_i)]$, where $\sum_{i=1}^r [G:C_G(g_i)] =14$ and $[G:C_G(g_i)] \in \{3,5\}$ for $i \in [r]$.\\
    Let $m$ and $n$ be the number of order 3 and order 5 conjugacy classes in $G$, respectively.\\
    Then $3m+5n =14$, where $m,n \in \bbn \cup \{0\}$.\\
    So
    \[3m = 14-5n \geq 0.\]
    So the possible $n$ can only be $0$ or $1$ or $2$.\\
    To make $3|(14-5n)$, just $n=1$ is satisfied and then $m = 3$.\\
	So we have $3+3+3+5 = 14$ and then $r= 4$.\\ 
    Since by class equation, we can just find one $r=4$ and corresponding $[G:C_G(g_i)]$ for $i = 1,2,3,4$, there is at most one possible class equation for $G$.




\end{proof}

\end{exer}

\end{document}

















