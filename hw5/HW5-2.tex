\documentclass{amsart}

% PACKAGES

\usepackage{amsmath}
\usepackage{amsfonts}
\usepackage{amssymb,enumerate}
\usepackage{amsthm,stmaryrd}
\usepackage[all]{xy}
\usepackage{hyperref}

%\theoremstyle{definition}
%\newtheorem{exer}{Exercise}

\newcommand{\bbr}{\mathbb{R}}
\newcommand{\bbc}{\mathbb{C}}
\newcommand{\bbz}{\mathbb{Z}}
\newcommand{\bbq}{\mathbb{Q}}
\newcommand{\bbn}{\mathbb{N}}
\newcommand{\be}{\mathbf{e}}
\newcommand{\ba}{\mathbf{a}}
\newcommand{\fm}{\mathfrak{m}}
\newcommand{\Hom}{\operatorname{Hom}}
\renewcommand{\ker}{\operatorname{Ker}}
\newcommand{\im}{\operatorname{Im}}
\newcommand{\xra}{\xrightarrow}
\newcommand{\wti}{\widetilde}

\theoremstyle{plain}
\newtheorem{lem}{Lemma}
\newtheorem{cor}[lem]{Corollary}
\newtheorem{prop}[lem]{Proposition}
\newtheorem{thm}[lem]{Theorem}
\newtheorem{conj}[lem]{Conjecture}
\newtheorem{intthm}{Theorem}
\renewcommand{\theintthm}{\Alph{intthm}}

\theoremstyle{definition}
\newtheorem{defn}[lem]{Definition}
\newtheorem{ex}[lem]{Example}
\newtheorem{question}[lem]{Question}
\newtheorem{questions}[lem]{Questions}
\newtheorem{problem}[lem]{Problem}
\newtheorem{disc}[lem]{Remark}
\newtheorem{rmk}[lem]{Remark}
\newtheorem{construction}[lem]{Construction}
\newtheorem{notn}[lem]{Notation}
\newtheorem{fact}[lem]{Fact}
\newtheorem{para}[lem]{}
\newtheorem{exer}[lem]{Exercise}
\newtheorem{remarkdefinition}[lem]{Remark/Definition}
\newtheorem{notation}[lem]{Notation}
\newtheorem{step}{Step}
\newtheorem{convention}[lem]{Convention}
\newtheorem*{Convention}{Convention}
\newtheorem{assumption}[lem]{Assumption}

\newcommand{\fmn}{F^{m\times n}}
\newcommand{\fnn}{F^{n\times n}}
\newcommand{\col}{\operatorname{Col}}
\newcommand{\row}{\operatorname{Row}}
\newcommand{\Span}{\operatorname{Span}}	
\newcommand{\rank}{\operatorname{rank}}	
\begin{document}

\noindent MATH 8510, Abstract Algebra I \\
Fall 2016\\
Exercises 5-2\\
Due date Thu 22 Sep 4:00PM

\

%\noindent
%Throughout this homework set, let $F$ be a field
%
%
%\

\begin{exer}[3.4.4]
Let $G$ be a finite abelian group, and let $n$ be a divisor of $|G|$. 
Use Cauchy's Theorem to show that $G$ has a subgroup of order $n$.
\begin{proof}
	\begin{enumerate}[(1)]
		\item
		  If $|G|=1$, then $G = \{e_G\}$ and only $1$ is a divisor of $|G|$.\\
		  So $G$ with order $1$ is a subgroup of $G$.
		 \item
		 Consider $1 < |G| < \infty$.\\
		 We will show it by induction.\\
		 Since $|G|\geq 1 and |G|\in \bbn$, $|G|$ can be written as $p_1p_2...p_{k_p}$, where $p_1,p_2,...p_{k_p}$ are primes.
		 \begin{enumerate}[(a)]
		   \item
             If $|G| = p$ is a prime, then the divisor $n$ can only be $1$ or $p$.\\
             So when $n = 1$, just let $H = \{e_G\} \leq G$; when $n = p$, we let $H = G\leq G$.\\
             Namely, it holds for the basic case.
		 	 \item
		 	  Suppose every group $G$ with $|G|= p_1,p_2,...p_{m}$ has the property that if $n$ is a divisor of $|G|$, then there exists a subgroup $H \leq G$ with $|H|=n$.
		 	  \item
				Let $G$ be a group such that $|G|= p_1,p_2,...p_{k_p}p_{m + 1}$. \\
				If $n=1$, just let $H = {e_G} \leq G$.\\
				If $n > 1$, then withour loss of generality, we assume $p_1|n$.\\
				By Cauchy's theorem, there exists $x \in G$ with $|x| = p_1$. \\
				Since $G$ is abelian, we have $\langle x\rangle$ is normal.\\
				Then $\langle x\rangle \unlhd G$ since $x\in G$.\\
				Then $|G/\langle x\rangle| = \frac{|G|}{|\langle x\rangle|} = \frac{p_1,p_2,...p_{m}}{p_1} = p_2,...p_{m+1}$.\\
				Since $n$ is a divisor of $|G|= p_1,p_2,...p_{k_p      }p_{m + 1}$ and $p_1|d$, then $\frac{n}{p_1}$ is a divisor of $p_2,p_3,...p_{m+1}$.\\
			   Thus, by assumption, there exists a subgroup $H/\langle x \rangle$ of $G/\langle x \rangle$ with order $|H/\langle x \rangle| = \frac{n}{p_1}$.\\
			   Besides, by the fourth isomophism theorem, we have $H \leq G$.\\ 
			   As last, by Lagrange's theorem, we have $|H| = |H/\langle x \rangle||\langle x \rangle|= \frac{n}{p_1} p_1 = n$.\\
			   Therefore, our assumption also holds for $|G|= p_1,p_2,...p_{k_p}p_{m + 1}$.
		\end{enumerate}
		Hence, the claim holds.\\
  \end{enumerate}
\end{proof}

\end{exer}

\begin{exer}
Let $G$ be a finite abelian group, written additively.
Prove that there is a chain of subgroups $0=N_0\unlhd N_1\unlhd\cdots\unlhd N_k=G$ such that each quotient $N_i/N_{i-1}$ is cyclic of prime order.
\begin{proof}
	Since $G$ is a finite group, then by Jordan-H\"{o}lder theorem, $G$ has a composition series.\\
	Namely, there is a chain of subgroups $0= N_0 \unlhd N_1 \unlhd...\unlhd N_k = G$.\\
	Then we show $N_n/N_{n-1}, n = 1,2,..k$ are abelian. \\
  	Let $gN_{n-1}, hN_{n-1} \in N_n/N_{n-1}$, where $g,h \in N_n$.\\
  	Since $N_{n-1} \unlhd N_{n} \unlhd G$ for $n = 1,2,...k$ and $G$ is abelian,\\
  	\begin{align*}
  	  gN_{n-1}hN_{n-1} &= ghN_{n-1} \\
  	  				&= hgN_{n-1}\\
  	  				&=hN_{n-1}gN_{n-1}
  	\end{align*}
  	So $N_n/N_{n-1}, n = 1,2,..k$ are abelian.\\
	Assume there exsit $n \in \bbz$ and $1 \leq n \leq k$ such that $N_n/N_{n-1}$ is of prime order.\\
	Then $|N_n/N_{n-1}| = p_1p_2..p_m$, where $p_1,p_2,...,p_m$ are primes and $ 1< m < \infty, m \in \bbn$.\\
	By Exercise 1, we know there exists $H \leq N_n/N_{n-1}$ with order $|H|=p_1$. \\
	Since $|\{e_{N_n/N_{n-1}}\}|=1 < p_1 < m = |N_n/N_{n-1}|$.\\
	Thus, ${e_{e_{N_n/N_{n-1}}}} \leq H \leq N_n/N_{n-1}$.\\ 
	So we have $N_n/N_{n-1}$ is not simple, which is contradicted by the chain $0=N_0\unlhd N_1\unlhd\cdots\unlhd N_k=G$ is a composition series.\\
	Therefore, each quotient $N_i/N_{i-1}$ for $i=1,2,..,n$ is of prime order.\\
	We have shown in Exercise 1(b) of homework 5-1, that $G$ is cyclic if the order of $p$ is a prime.\\
	As a result, each quotient $N_i/N_{i-1}$ for $i=1,2,..,n$ is of prime order
 \end{proof}
\end{exer}



\end{document}

















