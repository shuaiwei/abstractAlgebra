\documentclass{amsart}

% PACKAGES

\usepackage{amsmath}
\usepackage{amsfonts}
\usepackage{amssymb,enumerate}
\usepackage{amsthm,stmaryrd}
\usepackage[all]{xy}
\usepackage{hyperref}

%\theoremstyle{definition}
%\newtheorem{exer}{Exercise}

\newcommand{\bbr}{\mathbb{R}}
\newcommand{\bbc}{\mathbb{C}}
\newcommand{\bbz}{\mathbb{Z}}
\newcommand{\bbq}{\mathbb{Q}}
\newcommand{\bbn}{\mathbb{N}}
\newcommand{\be}{\mathbf{e}}
\newcommand{\ba}{\mathbf{a}}
\newcommand{\fm}{\mathfrak{m}}
\newcommand{\Hom}{\operatorname{Hom}}
\renewcommand{\ker}{\operatorname{Ker}}
\newcommand{\im}{\operatorname{Im}}
\newcommand{\xra}{\xrightarrow}
\newcommand{\wti}{\widetilde}

\theoremstyle{plain}
\newtheorem{lem}{Lemma}
\newtheorem{cor}[lem]{Corollary}
\newtheorem{prop}[lem]{Proposition}
\newtheorem{thm}[lem]{Theorem}
\newtheorem{conj}[lem]{Conjecture}
\newtheorem{intthm}{Theorem}
\renewcommand{\theintthm}{\Alph{intthm}}

\theoremstyle{definition}
\newtheorem{defn}[lem]{Definition}
\newtheorem{ex}[lem]{Example}
\newtheorem{question}[lem]{Question}
\newtheorem{questions}[lem]{Questions}
\newtheorem{problem}[lem]{Problem}
\newtheorem{disc}[lem]{Remark}
\newtheorem{rmk}[lem]{Remark}
\newtheorem{construction}[lem]{Construction}
\newtheorem{notn}[lem]{Notation}
\newtheorem{fact}[lem]{Fact}
\newtheorem{para}[lem]{}
\newtheorem{exer}[lem]{Exercise}
\newtheorem{remarkdefinition}[lem]{Remark/Definition}
\newtheorem{notation}[lem]{Notation}
\newtheorem{step}{Step}
\newtheorem{convention}[lem]{Convention}
\newtheorem*{Convention}{Convention}
\newtheorem{assumption}[lem]{Assumption}

\newcommand{\fmn}{F^{m\times n}}
\newcommand{\fnn}{F^{n\times n}}
\newcommand{\col}{\operatorname{Col}}
\newcommand{\row}{\operatorname{Row}}
\newcommand{\Span}{\operatorname{Span}}	
\newcommand{\rank}{\operatorname{rank}}	
\begin{document}

\noindent MATH 8510, Abstract Algebra I \\
Fall 2016\\
Exercises 2-2\\
Shuai Wei
\

%\noindent
%Throughout this homework set, let $F$ be a field
%
%
%\

\begin{exer}[1.6.20]
Let $G$ be a group, and let $\operatorname{Aut}(G)$ be the set of all isomorphisms $G\xra\cong G$.
Prove that $\operatorname{Aut}(G)$ is a group under function composition.
(The elements of $\operatorname{Aut}(G)$ are \emph{automorphism} of $G$, and $\operatorname{Aut}(G)$ is 
the \emph{automorphism group of $G$}.)
\begin{proof}
	$\forall g,h,k \in \operatorname{Aut}(G)$,
	\begin{enumerate}
			\item 
				$g: G \xra\cong G, h: G \xra\cong G$, then $gh: G \xra\cong G$ since $\cong$ is an equivalent relation.\\
				So $g\circ h \in \operatorname{Aut}(G)$.
			\item 
				Define $e: G \xra\cong G$ by $e(x) := x$. \\ 
				Then $e \in \operatorname{Aut}(G)$.\\
				Define $h: G \xra\cong G$ by $h(x) = y_x$.\\ 
				$\forall x \in G,$ we have $h\circ e(x) = hx$, so $h\circ e = e$.\\
				and $e\circ h(x) = e(y_x) = y_x = h(x)$, so $e\circ h = h$.\\
				Thus $\exists \; e \in \operatorname{Aut}(G)$ such that $e\circ h = e = h\circ e$.
			\item
				$\forall x \in G$, \\
				$g\circ (h\circ k)(x) = g\circ h\circ k(x) = (g\circ h)\circ k(x)$ by the definition of function composition.\\
				So $g\circ (h\circ k) = (g\circ h)\circ k$.
			\item
				$\forall x \in G, h(x)= y_x$ using the definition of $h$ from $(2)$.\\
				Define $h^{-1}: G \xra\cong G$ by $h^{-1}(y_x) := x$. \\
				Then $h^{-1} \in \operatorname{Aut}(G)$.\\
				So $h^{-1}\circ h(x) = h^{-1}(y_x) = x = e(x)$, so $h^{-1}\circ h = e$\\
				and $h\circ h^{-1}(y_x) = h(x) = y_x = e(y_x)$, so $h\circ h^{-1} = e$.\\
				Thus, $\exists\; h^{-1} \in \operatorname{Aut}(G)$ such that $h\circ h^{-1} = e = h^{-1}\circ h$.

	\end{enumerate}
\end{proof}


\end{exer}

\begin{exer}[1.7.16]
Let $G$ be a group. 
\begin{enumerate}[(a)]
\item
Prove that the formula $g\cdot x:=gxg^{-1}$ defines a group action of $G$ on itself. 
(This action is called the \emph{conjugation}. You may wish to compare it to similarity from linear algebra.)
\begin{proof}
	$\forall h, g, a, \in G, x \in G$,\\
	let $e_{G} \in G$ be the identity and then $e_{G}g = e_{G} = ge_{G}$.\\
	Then
	\begin{enumerate}
		\item
			$e_G\cdot a = e_{G}^{-1}ae_{G} = e_Gae_G= a$. 
		\item 
			$h\cdot (g\cdot a) =h\cdot (gag^{-1}) = h(gag^{-1})h^{-1} = hgag^{-1}h^{-1};$\\
			$(hg)\cdot a = hga(hg)^{-1} = hgag^{-1}h^{-1}.$\\
			So $h\cdot(g\cdot a) = (hg)\cdot a$.
	\end{enumerate}
\end{proof}


\item
For each $g\in G$, define $\sigma_g\colon G\to G$ by the formula $\sigma_g(x):=gxg^{-1}$. 
Prove that the rule $g\mapsto\sigma_g$ defines a group homomorphism $G\to\operatorname{Aut}(G)$.
(An automorphism of the form $\sigma_g$ using the conjugation action is called an \emph{inner} automorphism.)
\begin{proof}
	$\forall g,h,a\in G$, \\
	let $f$ be defined as $f: G\to\operatorname{Aut}(G)$ with the rule $g\mapsto\sigma_g$.\\
	Then $f(g) = \sigma_g$ and \\
	$f(gh) = \sigma_{gh}$ since $gh \in G$\\
	$f(gh)(a) = \sigma_{gh}(a) = (gh)a(gh)^{-1} = ghag^{-1}h^{-1}$.\\
	$f(g)\circ f(h)(a) = \sigma_{g}\circ \sigma_{h}(a) = \sigma_{g}(hah^{-1}) = g(hah^{-1})g^{-1} = ghah^{-1}g^{-1}$.\\
	So $\forall a\in G$, we have $f(gh)(a) = f(g)\circ f(h)(a)$.\\
	Thus, $f(gh) = f(g)\circ f(h)$.\\
	As a result, $f$ defines a group homomorphism.



\end{proof}
\end{enumerate}
\end{exer}

\begin{exer}
In your free time, read the statements of the following exercises.
\begin{enumerate}[1.2:]
\item[1.1:] 19, 20, 22--25, 31--36
\item[1.6:] 1--6, 8, 10--12, 17--23
\item[1.7:] 1--3, 8--10, 14--18
\end{enumerate}
\end{exer}


\end{document}


\begin{exer}
\begin{enumerate}[(a)]
\item 
\end{enumerate}
\end{exer}















