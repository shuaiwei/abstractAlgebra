\documentclass{amsart}

% PACKAGES

\usepackage{amsmath}
\usepackage{amsfonts}
\usepackage{amssymb,enumerate}
\usepackage{amsthm,stmaryrd}
\usepackage[all]{xy}
\usepackage{hyperref}

%\theoremstyle{definition}
%\newtheorem{exer}{Exercise}

\newcommand{\bbr}{\mathbb{R}}
\newcommand{\bbc}{\mathbb{C}}
\newcommand{\bbz}{\mathbb{Z}}
\newcommand{\bbq}{\mathbb{Q}}
\newcommand{\bbn}{\mathbb{N}}
\newcommand{\be}{\mathbf{e}}
\newcommand{\ba}{\mathbf{a}}
\newcommand{\fm}{\mathfrak{m}}
\newcommand{\Hom}{\operatorname{Hom}}
\renewcommand{\ker}{\operatorname{Ker}}
\newcommand{\im}{\operatorname{Im}}
\newcommand{\xra}{\xrightarrow}
\newcommand{\wti}{\widetilde}

\theoremstyle{plain}
\newtheorem{lem}{Lemma}
\newtheorem{cor}[lem]{Corollary}
\newtheorem{prop}[lem]{Proposition}
\newtheorem{thm}[lem]{Theorem}
\newtheorem{conj}[lem]{Conjecture}
\newtheorem{intthm}{Theorem}
\renewcommand{\theintthm}{\Alph{intthm}}

\theoremstyle{definition}
\newtheorem{defn}[lem]{Definition}
\newtheorem{ex}[lem]{Example}
\newtheorem{question}[lem]{Question}
\newtheorem{questions}[lem]{Questions}
\newtheorem{problem}[lem]{Problem}
\newtheorem{disc}[lem]{Remark}
\newtheorem{rmk}[lem]{Remark}
\newtheorem{construction}[lem]{Construction}
\newtheorem{notn}[lem]{Notation}
\newtheorem{fact}[lem]{Fact}
\newtheorem{para}[lem]{}
\newtheorem{exer}[lem]{Exercise}
\newtheorem{remarkdefinition}[lem]{Remark/Definition}
\newtheorem{notation}[lem]{Notation}
\newtheorem{step}{Step}
\newtheorem{convention}[lem]{Convention}
\newtheorem*{Convention}{Convention}
\newtheorem{assumption}[lem]{Assumption}

\newcommand{\fmn}{F^{m\times n}}
\newcommand{\fnn}{F^{n\times n}}
\newcommand{\col}{\operatorname{Col}}
\newcommand{\row}{\operatorname{Row}}
\newcommand{\Span}{\operatorname{Span}}	
\newcommand{\rank}{\operatorname{rank}}	
\begin{document}

\noindent MATH 8510, Abstract Algebra I \\
Fall 2016\\
Exercises 3-1\\
Shuai Wei
Collaborators: Liu XiaoYuan
\

%\noindent
%Throughout this homework set, let $F$ be a field
%
%
%\

\begin{exer}[2.1.8]
Let $H$ and $K$ be subgroups of a group $G$.
Prove that $H\cup K\leq G$ if and only if $H\subseteq K$ or $K\subseteq H$.

\begin{proof}
	\begin{enumerate}[(a)]
	\item 
		Assume $H\subseteq K$ or $K\subseteq H$.\\
		Then for convenience, we can suppose $H\subseteq K$. \\
		$e_G \in H \subset H \cup K$. So $H\cup K$ is not empty. \\
		$\forall x,y\in H \cup K$, we have $x, y \in H$ or $K$.\\
		Since $H \subseteq K$, we have $x,y \in K \subset H\cup K$.\\
		$K$ is a group, so $xy^{-1} \in K \subset H\cup K$.\\
		Thus, $H\cup K \leq G$.
	\item 
		Assume $H\cup K\leq G$. \\
		$\forall x \in H \subset H\cup K , y \in G \subset H\cup K$.\\
		we have $xy^{-1} \in H\cup K\leq G$ since $H\cup K$ is a subgroup. \\
		Then $xy^{-1}\in H$ or $K$.
		\begin{enumerate}[(1)]
			\item If $xy^{-1} \in H, x^{-1} \in H , x^{-1}x^y{-1} = y^{-1} \in H$. \\
				Then $(y^{-1})^{-1} = y\in H$.\\
				So $\forall y \in K$, we have $y \in H$.\\
				Thus, $K \subseteq H$.
			\item If $xy^{-1} \in K, y \in K, x^y{-1}y = x \in K$\\
				So $\forall x \in H$, we have $x \in K$.\\
				Thus, $H \subseteq K$.
		\end{enumerate}
		Hence, $H\subseteq K$ or $K\subseteq H$.
	\end{enumerate}
\end{proof}



\end{exer}

\begin{exer}[2.2.8]
Fix $i\in [n]=\{1,\ldots,n\}$ and set 
$G_i:=\{\sigma\in S_n\mid\sigma(i)=i\}$.
(In other words, $G_i$ is the stabilizer of $i$ in $G=S_n$.)
Prove that $S_{n-1}\cong G_i\leq S_n$.
\begin{proof}
	$ $\newline
	\begin{enumerate}
		\item
		Since $S_i$ is a stabilizer of $i$ in $S_n$, \\
		$G_i$ consists of all of the permutations of $\{i,1,2,...i-1,i+1,...,n\}$ with $i$ fixed. \\
		Namely, $G_i$ is all the permutations of $\{1,2,...,i-1,i+1,...n\}$.\\
		So it is obvious that $G_i$ is isomorphic to $S_{n-1}$.
		\item
	    For any element in $S_{n-1}$, it is a permutation in $S_n$ with $n$ fixed, \\
		so $G_i \subset S_n$.\\
		Besides, both of $S_{n-1}$ and $S_n$ are symmetric groups, which has same group operation, i.e, permutation composition.\\
		So $S_{n-1} \leq S_n$. \\
		\item
		By the definition of $G_i$, we have $G_i \subset S_n$.\\
		Let $e_{S_n}$ be the permutation which fixes $1,2,...n$, then $e_{S_n} \in S_{n}$ and $e_{S_n} \in G_{i}$ since it also fixes $i$.\\
		So $G_i \neq \emptyset$. \\
		$\forall x,y \in G_i \leq S_n, xy^{-1} \in S_n$ since $S_n$ is a group.\\
		$y^{-1} \in G_i$ since $y^{-1}$ just the reverse order of the permutation $y$, which also keeps $i$ fixed. \\
		So $xy^{-1}(i) = x(y^{-1}(i)) = x(i) = i$. \\
		Thus, $xy^{-1} \in G_i$.\\
		Hence $G_i \leq S_n$.

	\end{enumerate}
	

\end{proof}




\end{exer}

\begin{exer}
In your free time, read the statements of the following exercises.
\begin{enumerate}[1.2:]
\item[2.1:] 1, 4, 6--8, 10--13, 15--17
\item[2.2:] 1--3, 5(a), 6, 8--11
\end{enumerate}
\end{exer}


\end{document}


\begin{exer}
\begin{enumerate}[(a)]
\item 
\end{enumerate}
\end{exer}















