\documentclass{amsart}

% PACKAGES

\usepackage{amsmath}
\usepackage{amsfonts}
\usepackage{amssymb,enumerate}
\usepackage{amsthm,stmaryrd}
\usepackage[all]{xy}
\usepackage{hyperref}

%\theoremstyle{definition}
%\newtheorem{exer}{Exercise}

\newcommand{\bbr}{\mathbb{R}}
\newcommand{\bbc}{\mathbb{C}}
\newcommand{\bbz}{\mathbb{Z}}
\newcommand{\bbq}{\mathbb{Q}}
\newcommand{\bbn}{\mathbb{N}}
\newcommand{\be}{\mathbf{e}}
\newcommand{\ba}{\mathbf{a}}
\newcommand{\fm}{\mathfrak{m}}
\newcommand{\Hom}{\operatorname{Hom}}
\renewcommand{\ker}{\operatorname{Ker}}
\newcommand{\im}{\operatorname{Im}}
\newcommand{\xra}{\xrightarrow}
\newcommand{\wti}{\widetilde}

\theoremstyle{plain}
\newtheorem{lem}{Lemma}
\newtheorem{cor}[lem]{Corollary}
\newtheorem{prop}[lem]{Proposition}
\newtheorem{thm}[lem]{Theorem}
\newtheorem{conj}[lem]{Conjecture}
\newtheorem{intthm}{Theorem}
\renewcommand{\theintthm}{\Alph{intthm}}

\theoremstyle{definition}
\newtheorem{defn}[lem]{Definition}
\newtheorem{ex}[lem]{Example}
\newtheorem{question}[lem]{Question}
\newtheorem{questions}[lem]{Questions}
\newtheorem{problem}[lem]{Problem}
\newtheorem{disc}[lem]{Remark}
\newtheorem{rmk}[lem]{Remark}
\newtheorem{construction}[lem]{Construction}
\newtheorem{notn}[lem]{Notation}
\newtheorem{fact}[lem]{Fact}
\newtheorem{para}[lem]{}
\newtheorem{exer}[lem]{Exercise}
\newtheorem{remarkdefinition}[lem]{Remark/Definition}
\newtheorem{notation}[lem]{Notation}
\newtheorem{step}{Step}
\newtheorem{convention}[lem]{Convention}
\newtheorem*{Convention}{Convention}
\newtheorem{assumption}[lem]{Assumption}

\newcommand{\fmn}{F^{m\times n}}
\newcommand{\fnn}{F^{n\times n}}
\newcommand{\col}{\operatorname{Col}}
\newcommand{\row}{\operatorname{Row}}
\newcommand{\Span}{\operatorname{Span}}	
\newcommand{\rank}{\operatorname{rank}}	
\begin{document}

\noindent MATH 8510, Abstract Algebra I \\
Fall 2016\\
Exercises 3-2\\
Due date Thu 08 Sep 4:00PM

\

%\noindent
%Throughout this homework set, let $F$ be a field
%
%
%\

\begin{exer}[2.3.18--19 $+5\epsilon$]
Let $G$ be a group, and let $g\in G$.
\begin{enumerate}[(a)]
\item Prove that there exists a unique group homomorphism $f_g\colon\bbz\to G$ such that $f_g(1)=g$. 
	\begin{proof}
		Define
		\begin{align*}
			\bbz & \to G \\
			f_g(n) & \mapsto g^n
		\end{align*}
		Then $f_g$ is a homomorphism since $\forall m,n \in \bbz, f_g(m+n) = g^{m+n} = g^mg^n = f_g(m)f_g(n)$ and satisfies $f_g(1) = g$.\\
		So there exists a group homomorphism $f_g\colon\bbz\to G$ such that $f_g(1)=g$.\\
		Next we show it is unique.\\
		Assume there exists another homomorphism $h_g$ differing from $f_g$ such that $h_g(1) = g$. 
		\begin{enumerate}
			\item If $n > 0$, $h_g(n) = h_g(\sum_{i=1}^n1) = \prod_{i=1}^nh_g(1) = \prod_{i=1}^n g = g^n$ since $f_g$ is homomorphism.
			\item If $n < 0$, then $-n > 0$ and $h_g(n) = h_g(-(-n)) = (h_g(-n))^{-1} = (g^{-n})^{-1} = g^n$.
			\item If $n = 0$, then $h_g(0) = h_g(n-n) = h_g(n)h_g(-n) = g^ng^{-n} = g^{n-n} = g^0$.
				So $h_g(n) = g^n = f_g(n), \forall n \in \bbz$. \\
		\end{enumerate}
		Thus, $f_g = h_g$, which is contradicted by the assumption.\\
		Hence, such group homomorphism is unique.
	\end{proof}
	
\item Prove that $\im(f_g)=\langle g\rangle$.
	\begin{proof}
		$ $\newline
		$Im(f_g) = \{g^n | n \in \bbz\} \subset G$.\\
		So $g^n \in G, \forall n \in \bbz$.\\
		Thus, $<g> = \{g^n \in G | n \in \bbz\} = \{g^n | n \in \bbz\} = Im(f_g)$.
	\end{proof}
\item Prove that $f_g$ is a monomorphism if and only if $|g|=\infty$.
	\begin{proof}
		$ $\newline
		\begin{enumerate}
			\item
				Assume $f_g$ is a monomorphism, then $f_g$ is 1-1 since $f_g$ is homomorphism.\\
				Then $\infty= |\bbz| = |Im(f_g)| = |<\langle g \rangle| = |g|$.\\
				So $|g| = \infty$.
			\item 
				Assume $|g| = \infty$.\\
				Suppose $f_g$ is not homomorphism, then $f_g$ is not 1-1.\\
				So $\exists$ different $m,n \in \bbz$ such that $f_g(m) = g^m = g^n= f_g(n)$.\\
				Then $g^(m-n) = e_G$.\\
				So $|g| = |<g>| \leq m-n < \infty$ since $0< m-n < \infty$.\\
	
		\end{enumerate}


	\end{proof}

\item Assume that $|g|=n<\infty$.
\begin{enumerate}[(1)]
\item Prove that $\ker(f_g)=n\bbz:=\{nm\in\bbz\mid m\in\bbz\}$.
\item Prove that 
there is a unique group monomorphism $\phi_g\colon\bbz/n\bbz\to G$ such that $\phi_g(\overline 1)=g$. 
\item Prove that $\im(\phi_g)=\langle g\rangle$.
\item We say that a diagram 
of group homomorphisms
$$\xymatrix{A\ar[r]^-\alpha\ar[rd]_{\alpha'}&B\ar[d]^\beta \\ & C}$$
``commutes'' when $\beta\circ\alpha=\alpha'$.
Let $\pi\colon \bbz\to\bbz/n\bbz$ be the canonical epimorphism, and prove that the following diagram commutes.
$$\xymatrix{\bbz\ar[r]^-{\pi}\ar[rd]_{f_g}&\bbz/n\bbz\ar[d]^{\phi_g} \\ & G}$$
\end{enumerate}
\end{enumerate}
\end{exer}


\begin{exer}
In your free time, read the statements of the  exercises from Section~2.3.
\end{exer}


\end{document}


\begin{exer}
\begin{enumerate}[(a)]
\item 
\end{enumerate}
\end{exer}















