\documentclass{amsart}

% PACKAGES

\usepackage{amsmath}
\usepackage{amsfonts}
\usepackage{amssymb,enumerate}
\usepackage{amsthm,stmaryrd}
\usepackage[all]{xy}
\usepackage{hyperref}

%\theoremstyle{definition}
%\newtheorem{exer}{Exercise}

\newcommand{\bbr}{\mathbb{R}}
\newcommand{\bbc}{\mathbb{C}}
\newcommand{\bbz}{\mathbb{Z}}
\newcommand{\bbq}{\mathbb{Q}}
\newcommand{\bbn}{\mathbb{N}}
\newcommand{\be}{\mathbf{e}}
\newcommand{\ba}{\mathbf{a}}
\newcommand{\fm}{\mathfrak{m}}
\newcommand{\Hom}{\operatorname{Hom}}
\renewcommand{\ker}{\operatorname{Ker}}
\newcommand{\im}{\operatorname{Im}}
\newcommand{\xra}{\xrightarrow}
\newcommand{\wti}{\widetilde}

\theoremstyle{plain}
\newtheorem{lem}{Lemma}
\newtheorem{cor}[lem]{Corollary}
\newtheorem{prop}[lem]{Proposition}
\newtheorem{thm}[lem]{Theorem}
\newtheorem{conj}[lem]{Conjecture}
\newtheorem{intthm}{Theorem}
\renewcommand{\theintthm}{\Alph{intthm}}

\theoremstyle{definition}
\newtheorem{defn}[lem]{Definition}
\newtheorem{ex}[lem]{Example}
\newtheorem{question}[lem]{Question}
\newtheorem{questions}[lem]{Questions}
\newtheorem{problem}[lem]{Problem}
\newtheorem{disc}[lem]{Remark}
\newtheorem{rmk}[lem]{Remark}
\newtheorem{construction}[lem]{Construction}
\newtheorem{notn}[lem]{Notation}
\newtheorem{fact}[lem]{Fact}
\newtheorem{para}[lem]{}
\newtheorem{exer}[lem]{Exercise}
\newtheorem{remarkdefinition}[lem]{Remark/Definition}
\newtheorem{notation}[lem]{Notation}
\newtheorem{step}{Step}
\newtheorem{convention}[lem]{Convention}
\newtheorem*{Convention}{Convention}
\newtheorem{assumption}[lem]{Assumption}

\newcommand{\fmn}{F^{m\times n}}
\newcommand{\fnn}{F^{n\times n}}
\newcommand{\col}{\operatorname{Col}}
\newcommand{\row}{\operatorname{Row}}
\newcommand{\Span}{\operatorname{Span}}	
\newcommand{\rank}{\operatorname{rank}}	
\begin{document}

\noindent MATH 8510, Abstract Algebra I \\
Fall 2016\\
Exercises 4-1\\
Due date Thu 15 Sep 4:00PM

\

%\noindent
%Throughout this homework set, let $F$ be a field
%
%
%\


\begin{exer}
Let $G$ be a group. 
The \emph{commutator subgroup} of $G$ is the subgroup $[G,G]$ of $G$ generated by the set of all elements of the form $xyx^{-1}y^{-1}$:
$$[G,G]:=\langle xyx^{-1}y^{-1}\mid x,y\in G\rangle.$$
Let $f\colon G\to H$ be a group homomorphism.
\begin{enumerate}[(a)]
\item Prove that $A\subseteq G\implies f(\langle A\rangle)=\langle f(A)\rangle$.
	\begin{proof}
		$ $\newline
		First we show $f(\langle A\rangle)$ is a subgroup of $H$.\\
		$e_G \in \langle A\rangle$ since $\langle A \rangle$ is a subgroup of $G$.\\
		Then $f(e_G) = e_H \in f(\langle A \rangle)$ since f is homomorphism.\\
		So $f(\langle A\rangle) \neq \emptyset$.\\
		Let $y_1, y_2 \in f(\langle A\rangle)$. \\
		Then $\exists x_1, x_2 \in \langle A\rangle$ such that $f(x_1) = y_1 $ and $f(x_2) = y_2$.\\
		$\langle A\rangle$ is a subgroup of $G$, so $x_2^{-1} \in \langle A\rangle$.\\
		So $x_1x_2^{-1} \in \langle A\rangle$ and then $f(x_1x_2^{-1}) \in f(\langle A\rangle)$.\\
		Since $f$ is a homomorphism, \\
		$f(x_1x_2^{-1}) = f(x_1)f(x_2^{-1}) =y_1f(x_2)^{-1} = y_1y_2^{-1} \in f(\langle A\rangle)$.\\
		Besides, $f(\langle A\rangle) \subset H$.\\ 
		Thus, $f(\langle A\rangle)$ is a subgroup of $H$.\\
		$f(A) \subseteq f(\langle A\rangle)$, so 
		$$\langle f(A) \rangle \subseteq f(\langle A\rangle).$$
		Let $x \in \langle A \rangle$, then $x = a_1^{\epsilon_1}a_2^{\epsilon_2}...a_n^{\epsilon_n}$ for some $n \in \bbn$ and $a_1,a_2,....a_n \in A$ and $\epsilon_1,\epsilon_2,...\epsilon_n \in \bbz$.
		\begin{align*}
			f(x) &= f(a_1^{\epsilon_1}a_2^{\epsilon_2}...a_n^{\epsilon_n}) \\
				 	  &= {f(a_1)}^{\epsilon_1} {f(a_2)}^{\epsilon_2}...{f(a_n)}^{\epsilon_n}\\
				 	  & \in f(A) \subseteq \langle f(A)\rangle
		\end{align*}
		since $f$ is a homomorphism and $f(a_1)^{\epsilon_1},f(a_2)^{\epsilon_2},...,f(a_n)^{\epsilon_n} \in f(A)$.\\
		So we have 
		$$f(\langle A \rangle) \subseteq \langle f(A)\rangle.$$
		Since both of $f(\langle A\rangle)$ and $\langle f(A) \rangle$ are subgroups of $H$,\\
		$f(\langle A\rangle)=\langle f(A)\rangle$
	\end{proof}
\item Prove that $B\subseteq C\subseteq H\implies \langle B\rangle\subseteq\langle C\rangle$.
	\begin{proof}
		$ $\newline
		$B \subseteq C \subseteq H$, so $B\subseteq C \subseteq \langle C \rangle \subseteq H$.\\
		So $B\subseteq \langle C \rangle$.\\
		Thus,$\langle B \rangle \subseteq \langle C \rangle$ since $\langle C \rangle$ is a subgroup of $H$.
	\end{proof}

\item Prove that $G$ is abelian if and only if $[G,G]=\{e\}$.
	\begin{proof}
		$ $\newline
		\begin{enumerate}
			\item 
				Assume $G$ is abelian.\\
				Then 
				\begin{align*}
				[G,G] &= \langle xyx^{-1}y^{-1} |x,y \in G\langle \\
			    	  &= \langle (xx^{-1})(yy^{-1}) |x,y \in G \langle \\
			    	  &= \langle e |x,y \in G \langle \\
			    	  & = \{e\}
				\end{align*}	
			\item
		\end{enumerate}
	\end{proof}
\item Prove that $f([G,G])\subseteq[H,H]$.
\item Prove that if   $H$ is abelian, then $[G,G]\subseteq\ker(f)$.
\end{enumerate}
\end{exer}

\begin{exer}[2.4.14]
See the text for a hint for this exercise.
\begin{enumerate}[(a)]
\item Prove that every finitely generated subgroup of $(\bbq,+)$ is cyclic.
\item Prove that $(\bbq,+)$ is not cyclic.
Conclude that $(\bbq,+)$ is not finitely generated.
\end{enumerate}
\end{exer}



\end{document}


\begin{exer}
\begin{enumerate}[(a)]
\item 
\end{enumerate}
\end{exer}















