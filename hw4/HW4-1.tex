\documentclass{amsart}

% PACKAGES

\usepackage{amsmath}
\usepackage{amsfonts}
\usepackage{amssymb,enumerate}
\usepackage{amsthm,stmaryrd}
\usepackage[all]{xy}
\usepackage{hyperref}

%\theoremstyle{definition}
%\newtheorem{exer}{Exercise}

\newcommand{\bbr}{\mathbb{R}}
\newcommand{\bbc}{\mathbb{C}}
\newcommand{\bbz}{\mathbb{Z}}
\newcommand{\bbq}{\mathbb{Q}}
\newcommand{\bbn}{\mathbb{N}}
\newcommand{\be}{\mathbf{e}}
\newcommand{\ba}{\mathbf{a}}
\newcommand{\fm}{\mathfrak{m}}
\newcommand{\Hom}{\operatorname{Hom}}
\renewcommand{\ker}{\operatorname{Ker}}
\newcommand{\im}{\operatorname{Im}}
\newcommand{\xra}{\xrightarrow}
\newcommand{\wti}{\widetilde}

\theoremstyle{plain}
\newtheorem{lem}{Lemma}
\newtheorem{cor}[lem]{Corollary}
\newtheorem{prop}[lem]{Proposition}
\newtheorem{thm}[lem]{Theorem}
\newtheorem{conj}[lem]{Conjecture}
\newtheorem{intthm}{Theorem}
\renewcommand{\theintthm}{\Alph{intthm}}

\theoremstyle{definition}
\newtheorem{defn}[lem]{Definition}
\newtheorem{ex}[lem]{Example}
\newtheorem{question}[lem]{Question}
\newtheorem{questions}[lem]{Questions}
\newtheorem{problem}[lem]{Problem}
\newtheorem{disc}[lem]{Remark}
\newtheorem{rmk}[lem]{Remark}
\newtheorem{construction}[lem]{Construction}
\newtheorem{notn}[lem]{Notation}
\newtheorem{fact}[lem]{Fact}
\newtheorem{para}[lem]{}
\newtheorem{exer}[lem]{Exercise}
\newtheorem{remarkdefinition}[lem]{Remark/Definition}
\newtheorem{notation}[lem]{Notation}
\newtheorem{step}{Step}
\newtheorem{convention}[lem]{Convention}
\newtheorem*{Convention}{Convention}
\newtheorem{assumption}[lem]{Assumption}

\newcommand{\fmn}{F^{m\times n}}
\newcommand{\fnn}{F^{n\times n}}
\newcommand{\col}{\operatorname{Col}}
\newcommand{\row}{\operatorname{Row}}
\newcommand{\Span}{\operatorname{Span}}	
\newcommand{\rank}{\operatorname{rank}}	
\begin{document}

\noindent MATH 8510, Abstract Algebra I \\
Fall 2016\\
Exercises 4-1\\
Due date Thu 15 Sep 4:00PM\\
Shuai Wei
\

%\noindent
%Throughout this homework set, let $F$ be a field
%
%
%\


\begin{exer}
Let $G$ be a group. 
The \emph{commutator subgroup} of $G$ is the subgroup $[G,G]$ of $G$ generated by the set of all elements of the form $xyx^{-1}y^{-1}$:
$$[G,G]:=\langle xyx^{-1}y^{-1}\mid x,y\in G\rangle.$$
Let $f\colon G\to H$ be a group homomorphism.
\begin{enumerate}[(a)]
\item Prove that $A\subseteq G\implies f(\langle A\rangle)=\langle f(A)\rangle$.
	\begin{proof}
		$ $\newline
		First we show $f(\langle A\rangle)$ is a subgroup of $H$.\\
		$e_G \in \langle A\rangle$ since $\langle A \rangle$ is a subgroup of $G$.\\
		Then $f(e_G) = e_H \in f(\langle A \rangle)$ since f is a homomorphism.\\
		So $f(\langle A\rangle) \neq \emptyset$.\\
		Let $y_1, y_2 \in f(\langle A\rangle)$. \\
		Then $\exists x_1, x_2 \in \langle A\rangle$ such that $f(x_1) = y_1 $ and $f(x_2) = y_2$.\\
		$\langle A\rangle$ is a subgroup of $G$, so $x_2^{-1} \in \langle A\rangle$.\\
		So $x_1x_2^{-1} \in \langle A\rangle$ and then $f(x_1x_2^{-1}) \in f(\langle A\rangle)$.\\
		Since $f$ is a homomorphism, \\
		$f(x_1x_2^{-1}) = f(x_1)f(x_2^{-1}) =y_1f(x_2)^{-1} = y_1y_2^{-1} \in f(\langle A\rangle)$.\\
		Besides, $f(\langle A\rangle) \subset H$.\\ 
		Thus, $f(\langle A\rangle)$ is a subgroup of $H$.\\
		It is obvious that $f(A) \subseteq f(\langle A\rangle)$, \\
		so 
		$$\langle f(A) \rangle \subseteq f(\langle A\rangle).$$
		Let $x \in \langle A \rangle$, then $x = a_1^{\epsilon_1}a_2^{\epsilon_2}...a_n^{\epsilon_n}$ for some $n \in \bbn$ and $a_1,a_2,....a_n \in A$ and $\epsilon_1,\epsilon_2,...\epsilon_n \in \bbz$.\\
		Then
		\begin{align*}
			f(x) &= f(a_1^{\epsilon_1}a_2^{\epsilon_2}...a_n^{\epsilon_n}) \\
				 	  &= {f(a_1)}^{\epsilon_1} {f(a_2)}^{\epsilon_2}...{f(a_n)}^{\epsilon_n}\\
				 	  & \in f(A) \subseteq \langle f(A)\rangle
		\end{align*}
		since $f$ is a homomorphism and $f(a_1)^{\epsilon_1},f(a_2)^{\epsilon_2},...,f(a_n)^{\epsilon_n} \in f(A)$.\\
		So we have 
		$$f(\langle A \rangle) \subseteq \langle f(A)\rangle.$$
		Since both of $f(\langle A\rangle)$ and $\langle f(A) \rangle$ are subgroups of $H$,\\
		$f(\langle A\rangle)=\langle f(A)\rangle$
	\end{proof}
\item Prove that $B\subseteq C\subseteq H\implies \langle B\rangle\subseteq\langle C\rangle$.
	\begin{proof}
		$ $\newline
		$B \subseteq C \subseteq H$, so $B\subseteq C \subseteq \langle C \rangle \subseteq H$.\\
		So $B\subseteq \langle C \rangle$.\\
		Thus,$\langle B \rangle \subseteq \langle C \rangle$ since $\langle C \rangle$ is a subgroup of $H$.
	\end{proof}

\item Prove that $G$ is abelian if and only if $[G,G]=\{e\}$.
	\begin{proof}
		$ $\newline
		\begin{enumerate}
			\item 
				Assume $G$ is abelian.\\
				Then 
				\begin{align*}
				[G,G] &= \langle xyx^{-1}y^{-1} |x,y \in G\rangle \\
			    	  &= \langle (xx^{-1})(yy^{-1}) |x,y \in G \rangle \\
			    	  &= \langle e |x,y \in G \rangle \\
			    	  & = \{e\}
				\end{align*}	
			\item
				Assume $[G,G] = \{e\}$.\\ 
				By the definition of $[G,G]$, we have 
				$\forall x,y \in G$,
				\begin{align*}
					xyx^{-1}y^{-1} = e  & \Longleftrightarrow xyx^{-1} = y. \\
										& \Longleftrightarrow xy = yx. 
				\end{align*}
				So $G$ is abelian.
		\end{enumerate}
	\end{proof}
\item Prove that $f([G,G])\subseteq[H,H]$.
	\begin{proof}
		$ $\newline
		$\forall x,y \in G, xyx^{-1}y^{-1} \in [G,G]$,\\
		$f(xyx^{-1}y^{-1}) = f(x)f(y)f(x^{-1})f(y^{-1}) = f(x)f(y){f(x)}^{-1}{f(y)}^{-1}$. \\
		since $f$ is a homomorphism.\\
		Moreover, $f(x), f(y) \in H$, so $f(x)f(y){f(x)}^{-1}{f(y)}^{-1} \in [H,H]$.\\
		Thus, $f([G,G]) \subset [H,H]$.\\
	\end{proof}
\item Prove that if $H$ is abelian, then $[G,G]\subseteq\ker(f)$.
	\begin{proof}
		$ $\newline
		$\forall x,y \in G, xyx^{-1}y^{-1} \in [G,G]$,\\
		$f(xyx^{-1}y^{-1}) = f(x)f(y)f(x^{-1})f(y^{-1}) = f(x)f(y){f(x)}^{-1}{f(y)}^{-1}$. \\
		since $f$ is a homomorphism.\\
		Moreover, $f(x), f(y) \in H$ and $H$ is abelian. \\
		Then $f(xyx^{-1}y^{-1}) = \Big(f(x){f(x)}^{-1}\Big)\Big(f(y){f(y)}^{-1}\Big) = e_He_H = e_H$.\\
		Thus, $[G,G]\subseteq\ker(f)$.
	\end{proof}
\end{enumerate}
\end{exer}

\begin{exer}[2.4.14]
See the text for a hint for this exercise.
\begin{enumerate}[(a)]
\item Prove that every finitely generated subgroup of $(\bbq,+)$ is cyclic.
	\begin{proof}
		$ $\newline
		Assume $H$ is finitely generated subgroup of $\bbq$\\ 
		Since $\forall q \in \bbq, q = \frac{m}{n}$ for some $m,n\in \bbn$ and $n \neq 0$, $H$ can be written as 
		$$H = \Big\langle \frac{a_1}{b_1},\frac{a_2}{b_2},...,\frac{a_n}{b_n} \Big\rangle,$$ 
		where $n \in \bbn$ and $a_i,b_i \in \bbn $ for $i =1,2,...,n$. \\
		Since $(\bbq,+)$ is abelian and $\frac{a_i}{b_i} \in \bbq$ for $i = 1,2,...,n$, 
		\begin{align*}
			H &= \Big\langle \frac{a_1}{b_1},\frac{a_2}{b_2},...,\frac{a_n}{b_n} \Big\rangle \\
			  &= \Big\{\epsilon_1\big(\frac{a_1}{b_1}\big)+\epsilon_2\big(\frac{a_2}{b_2}\big) + ...+\epsilon_n\big(\frac{a_n}{b_n}\big)|\epsilon_1,\epsilon_2,...,\epsilon_n \in \bbz\Big\}. 
	    \end{align*}
		Let $k = b_1b_2...b_n \neq 0$ since $b_i \neq 0$ for $i =1,2,...,n$.\\
		We claim $H \leq \langle \frac{1}{k} \rangle$.\\
	    Let $h \in H$, then 
	    $$h = z_1(\frac{a_1}{b_1})+z_2(\frac{a_2}{b_2}) + ...+z_n(\frac{a_n}{b_n}) \text{ for some } z_1,z_2,...,z_n \in \bbz.$$
	    So 
	    $$h = \frac{z_1a_1b_2...b_n + z_2a_2b_1b_3...b_n + z_na_nb_1b_2...b_{n-1}}{b_1b_2...b_n} = \frac{z}{k},$$
	    where $z = z_1a_1b_2...b_n + z_2a_2b_1b_3...b_n + z_na_nb_1b_2...b_{n-1} \in \bbz$ since $z_i,a_i,b_i \in \bbz$ for $i = 1,2,...,n$.\\
	    Thus $h \in \langle \frac{1}{k} \rangle$.\\
	    Therefore $H \subseteq \langle \frac{1}{k} \rangle$.\\
	    As a result, our claim $H \leq \langle \frac{1}{k} \rangle$ holds since $H, \langle \frac{1}{k} \rangle $ are both groups.\\
	    It is obvious $\langle \frac{1}{k} \rangle$ is cyclic.\\
	    So $H$ is also cyclic.
	\end{proof}
\item Prove that $(\bbq,+)$ is not cyclic.
Conclude that $(\bbq,+)$ is not finitely generated.
	\begin{proof}
	$ $\newline
	Assume $(\bbq,+)$ is cyclic\\
	Then $\exists a,b \in \bbz$ and $b \neq 0$ such that $\bbq = \langle \frac{a}{b}\rangle$. \\
	Since $\frac{a}{2b} \in (\bbq,+)$, $\frac{a}{2b} = n\frac{a}{b}$ for some $n \in \bbz$.\\
	Then we have $n = \frac{1}{2}$, which is contradicted by $n\in \bbz$.\\
	Thus, $(\bbq,+)$ is not cyclic.\\
	Hence, we conclude that $(\bbq,+)$ is not finitely generated. Otherwise, by part $(a)$, we have $(\bbq,+)$ is cyclic, which is a contradiction since we have shown $(\bbq,+)$ is not cyclic.

	\end{proof}
\end{enumerate}
\end{exer}



\end{document}


\begin{exer}
\begin{enumerate}[(a)]
\item 
\end{enumerate}
\end{exer}















